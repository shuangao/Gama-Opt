The motivation of transfer knowledge between different domains is to apply the previous information from the source domain to the target one, assuming that there exists certain relationship, explicit or implicit, between the  feature space of these two domains \cite{pan2010survey}. Technically, previous work can be concluded into solving the following three issues: what, how and when to transfer \cite{tommasi2014learning}.


\textbf{What to transfer.} Previous work tried to answer this question from three different aspects: selecting transferable instances, learning transferable feature representations and transferable model parameters. Instance-based transfer learning assume that part of the instances in the source domain could be re-used to benefit the learning for the target domain. Lim et al. proposed a method of augmenting the training data by borrowing data from other classes for object detection \cite{lim2012transfer}. Learning transferable features means to learn common feature that can alleviate the bias of data distribution in target domain. Recently, Long et al. proposed a method that can learn transferable features with deep neural network and showed some impressive results on the  benchmarks \cite{LongICML15}. Parameter transfer
approach assumes that the parameters of the model for the source task can be transfered to the target task. Yang et al. proposed Adaptive SVMs by transferring parameters by incorporating the auxiliary classifier trained from source domain \cite{yang2007cross}. On top of Yang's work, Ayatar et al proposed PMT-SVM that can determine the transfer regularizer according to the target data automatically \cite{aytar2011tabula}. Tommasi et al. proposed Multi-KT that can utilize the parameters from multiple source models for the target classes  \cite{tommasi2014learning}.
Kuzborskij et al. proposed a similar method to learn new categories by leveraging over the known source \cite{kuzborskij2013n}.

\textbf{When and how to transfer.} The question \textit{when to transfer} arises when we want to know if the information acquired from previous task is relevant to the new one (i.e. in what situation, knowledge should not be transfered). In some extreme situation, where the source domain and target one are not relevant, brutal-force transferring the knowledge between them could even degrade the performance of the classifier in target domain. This is often referred to as negative transfer. \textit{How to transfer} the prior knowledge effectively should be carefully designed to prevent inefficient and negative transfer. Some previous work consists in using generative probabilistic method \cite{davis2009deep} \cite{wang2014active} \cite{zhou2014multi}.  Bayesian learning methods can predict the target domain by combining the prior source distribution to generate a posterior distribution. Alternatively, some previous max margin methods show that it is possible to learn from a few examples by minimizing the  Leave-One-Out (LOO) error for the training model \cite{kuzborskij2013n} \cite{tommasi2010safety}. Previous work shows that there is a closed-form implementation of LOO cross-validation that can generate unbiased model estimation for LS-SVM \cite{cawley2006leave}.

Our work correspond to the context above. \hl{In this paper, we propose a method based on parameter transfer approach with max margin classifier. We address our work on how to prevent negative transfer when we are not sure whether the prior knowledge is relevant to our target task. We mathematically proof that, without any data distrbution assumption, the superior bound of the training loss for our transfer method is the loss of a method learning directly (i.e. without using any prior knowledge). This indecates that when the prior knowledge hurts the transfer procedure, our method can avoid negative transfer automatically.} 
