The motivation of transfer knowledge between different domains is to apply the previous information from the source domain to the target one, assuming that there exists certain relationship, explicit or implicit, between the  feature space of these two domains \cite{pan2010survey}. Technically, previous work can be concluded into solving the following three issues: what, how and when to transfer \cite{tommasi2014learning}.


\textbf{What to transfer.} Previous work tried to answer this question from three different aspects: learning transferable instances, feature representations and model parameters. To utilize the data from previous classes, Lim et al. proposed a method of augmenting the training data by borrowing data from other classes for object detection \cite{lim2012transfer}. Learning transferable features means to learn common feature that can alleviate the bias of data distribution in target domain. Recently, Long et al. proposed a method that can learn transferable features with deep neural network  and showed some impressive results on some  benchmarks\cite{LongICML15}.Parameter transfer
approach assumes that the parameters of the model for the source task can be transfered to the target task. Yang et al. proposed Adaptive SVMs by transferring parameters from the auxiliary classifier from trained from source domain\cite{yang2007cross}. On top of Yang's work, Ayatar et al proposed PMT-SVM that can determine the transfer regularizer according to the target data automatically \cite{aytar2011tabula}. Tommasi et al. proposed Multi-KT that can utilize the parameters from multiple source models for the target classes  \cite{tommasi2014learning}.
Kuzborskij et al. proposed a similar method to learn new categories by leveraging over the known source \cite{kuzborskij2013n}.

\textbf{How to transfer.}
